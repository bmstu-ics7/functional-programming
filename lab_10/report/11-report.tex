\section{ЗАДАНИЕ 1}

Пусть list-of-list список, состоящий из списков.
Написать функцию, которая вычисляет сумму длин всех элементов
list-of-list.

\begin{lstlisting}
(defun sum_length (lst)
    (cond
        ((null lst) 0)
        (T (+ (length (car lst)) (sum_length (cdr lst))))
    )
)

(sum_length '((1 2 3) (1 2) (1) (1 2 3 4) (1 2))) ;;; 12
\end{lstlisting}

\section{ЗАДАНИЕ 2}

Написать рекурсивную версию (с именем reg-add) вычисления суммы чисел
заданного списка.

\begin{lstlisting}
(defun reg-add (lst)
    (cond
        ((null lst) 0)
        (T (+ (car lst) (reg-add (cdr lst))))
    )
)

(reg-add '(2 4 6)) ;;; 12
\end{lstlisting}

\section{ЗАДАНИЕ 3}

Написать рекурсивную версию с именем recnth функции nth.

\begin{lstlisting}
(defun recnth (n lst)
    (cond
        ((null lst) Nil)
        ((eql n 0) (car lst))
        (T (recnth (- n 1) (cdr lst)))
    )
)

(recnth 0 '(1 2 3)) ;;; 1
(recnth 1 '(1 2 3)) ;;; 2
(recnth 2 '(1 2 3)) ;;; 3
(recnth 3 '(1 2 3)) ;;; Nil
\end{lstlisting}

\section{ЗАДАНИЕ 4}

Написать рекурсивную функцию alloddr, которая возвращает t когда все
элементы списка нечетные.

\begin{lstlisting}
(defun alloddr (lst)
    (cond
        ((null lst) T)
        ((eql (mod (car lst) 2) 0) Nil)
        (T (alloddr (cdr lst)))
    )
)

(alloddr '(1 2 3 4)) ;;; Nil
(alloddr '(1 3 5 7 9 10)) ;;; Nil
(alloddr '(1 3 5 7 9)) ;;; T
\end{lstlisting}

\section{ЗАДАНИЕ 5}

Написать рекурсивную функцию, относящуюся к хвостовой рекурсии
с одним тестом завершения, которая возвращает последний элемент
списка-аргумента.

\begin{lstlisting}
(defun last_rec (lst)
    (cond
        ((null lst) Nil)
        ((eql (length lst) 1) (car lst))
        (T (last_rec (cdr lst)))
    )
)

(last_rec ()) ;;; Nil
(last_rec '(1 2 3 4)) ;;; 4
\end{lstlisting}

\section{ЗАДАНИЕ 6}

Написать рекурсивную функцию, относящуюся к дополняемой рекурсии с
одним тестом завершения, которая вычисляет сумму всех чисел
от 0 до n-ого аргумента функции.

Вариант:

\begin{enumerate}
    \item от п-аргумента функции до последнего >= 0,
    \item от п-аргумента функции до т-аргумента с шагом d.
\end{enumerate}

\begin{lstlisting}
(defun sum_recursive (a b d n lst)
    (cond
        ((null lst) 0)
        ((< b 0) 0)
        ((and (<= a 0) (eql n 0))
            (+
                (car lst)
                (sum_recursive a (- b 1) d (- d 1) (cdr lst)))
        )
        ((<= a 0) (sum_recursive a (- b 1) d (- n 1) (cdr lst)))
        (T (sum_recursive (- a 1) (- b 1) d 0 (cdr lst)))
    )
)

(defun sum_elements (a b d lst)
    (cond
        ((or (> a b) (<= d 0)) 0)
        (T (sum_recursive a b d 0 lst))
    )
)

(sum_elements 5 3 1 '(1 2 3 4 5 6 7 8 9 10)) ;;; 0
(sum_elements 0 9 1 '(1 2 3 4 5 6 7 8 9 10)) ;;; 1+2+3+4+5+6+7+8+9+10 = 55
(sum_elements 2 8 1 '(1 2 3 4 5 6 7 8 9 10)) ;;; 3+4+5+6+7+8+9 = 42
(sum_elements 2 5 2 '(1 2 3 4 5 6 7 8 9 10)) ;;; 3+5 = 8
(sum_elements 2 8 3 '(1 2 3 4 5 6 7 8 9 10)) ;;; 3+6+9 = 18
\end{lstlisting}

\section{ЗАДАНИЕ 7}

Написать рекурсивную функцию, которая возвращает последнее нечетное
число из числового списка, возможно создавая некоторые
вспомогательные функции.

\begin{lstlisting}
(defun last_odd (lst)
    (cond
        ((null lst) Nil)
        (T
            (let
                ((next_iteration (last_odd (cdr lst))))
                (cond
                    (
                        (and
                            (eql (mod (car lst) 2) 1)
                            (not next_iteration)
                        ) (car lst)
                    )
                    (T next_iteration)
                )
            )
        )
    )
)

(last_odd '(1 2 3 4 5 6)) ;;; 5
(last_odd '(1 2 3 4 5)) ;;; 5
(last_odd '(2 4 6 8)) ;;; Nil
\end{lstlisting}

\section{ЗАДАНИЕ 8}

Используя cons-дополняемую рекурсию с одним тестом завершения,
написать функцию которая получает как аргумент список чисел,
а возвращает список квадратов этих чисел в том же порядке.

\begin{lstlisting}
(defun square_list (lst)
    (cond
        ((null lst) Nil)
        (T (cons (* (car lst) (car lst)) (square_list (cdr lst))))
    )
)

(square_list '(1 2 3 4 5)) ;;; (1 4 9 16 25)
\end{lstlisting}

\section{ЗАДАНИЕ 9}

Написать функцию с именем select-odd, которая из заданного
списка выбирает все нечетные числа. (Вариант 1: select-even,
вариант 2: вычисляет сумму всех нечетных чисел(sum-all-odd)
или сумму всех четных чисел (sum-all-even) из заданного списка.)

\begin{lstlisting}
(defun select_odd_or_even (lst n)
    (cond
        ((null lst) Nil)
        ((eql (mod (car lst) 2) n)
            (append (list (car lst)) (select_odd_or_even (cdr lst) n))
        )
        (T (select_odd_or_even (cdr lst) n))
    )
)

(defun select-odd (lst) (select_odd_or_even lst 1))
(defun select-even (lst) (select_odd_or_even lst 0))

(select-odd '(1 2 3 4 5 6 7 8 9 10)) ;;; (1 3 5 7 9)
(select-even '(1 2 3 4 5 6 7 8 9 10)) ;;; (2 4 6 8 10)

(defun sum_odd_or_even (lst n)
    (cond
        ((null lst) 0)
        ((eql (mod (car lst) 2) n)
            (+ (car lst) (sum_odd_or_even (cdr lst) n))
        )
        (T (sum_odd_or_even (cdr lst) n))
    )
)

(defun sum-odd (lst) (sum_odd_or_even lst 1))
(defun sum-even (lst) (sum_odd_or_even lst 0))

(sum-odd '(1 2 3 4 5 6 7 8 9 10)) ;;; 25
(sum-even '(1 2 3 4 5 6 7 8 9 10)) ;;; 30
\end{lstlisting}

\section{ТЕОРЕТИЧЕСКИЕ ВОПРОСЫ}

\subsubsection{Способы организации повторных вычислений в Lisp}

\begin{itemize}
    \item использование функционалов
    \item использование рекурсии
\end{itemize}

\subsubsection{Что такое рекурсия?
Классификация рекурсивных функций в Lisp}

\textbf{Рекурсия} -- это ссылка на определяемый объект во
время его определения.
Т. к. в Lisp используются рекурсивно определенные структуры (списки),
то рекурсия -- это естественный принцип обработки таких структур.

\begin{itemize}
    \item простая рекурсия - один рекурсивный вызов в теле
    \item рекурсия первого порядка - рекурсивный вызов встречается несколько раз
    \item взаимная рекурсия - используется несколько функций,
        рекурсивно вызывающих друг друга.
\end{itemize}

\subsubsection{Различные способы организации
рекурсивных функций и порядок их реализации}

\begin{itemize}
    \item Хвостовая рекурсия. В целях повышения эффективности рекурсивных
        функций рекомендуется формировать результат не на выходе из рекурсии,
        а на входе в рекурсию, все действия выполняя до ухода на следующий шаг
        рекурсии. Это и есть хвостовая рекурсия.
    \item Возможна рекурсия по нескольким параметрам
    \item Дополняемая рекурсия -- при обращении к рекурсивной функции
        используется дополнительная функция не в аргументе вызова , а вне его
    \item Выделяют группу функций множественной рекурсии. На одной
        ветке происходит сразу несколько рекурсивных вызовов. Количество
        условий выхода также может зависеть от задачи.
\end{itemize}

\subsubsection{Способы повышения эффективности реализации рекурсии}

\begin{itemize}
    \item Использование хвостовой рекурсии.
        Если условий выхода несколько, то надо думать о порядке их следования.
    \item Превращение не хвостовой рекурсии в хвостовую.
        Для превращения не хвостовой рекурсии в хвостовую и в целях
        формирования результата (результирующего списка) на входе в
        рекурсию, рекомендуется использовать дополнительные (рабочие)
        параметры. При этом становится необходимым создат фунецию --
        оболочку для реализации очевидного обращения к функции.
\end{itemize}
