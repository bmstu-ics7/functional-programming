\section{ЗАДАНИЕ 1}

Написать функцию, которая принимает целое число и возвращает первое
четное число, не меньшее аргумента.

\begin{lstlisting}
(defun first_even_after (num)
    (if (equal (mod num 2) 0)
        num
        (+ num 1)
    )
)

(first_even_after 12) ;;; 12
(first_even_after 13) ;;; 14
\end{lstlisting}

\section{ЗАДАНИЕ 2}

Написать функцию, которая принимает число и возвращает число
того же знака, но с модулем на 1 больше модуля аргумента.

\begin{lstlisting}
(defun abs_plus_one (num)
    (/ (* (+ (abs num) 1) num) (abs num))
)

(defun abs_plus_one (num)
    (if (> num 0) (+ num 1) (- num 1))
)

(abs_plus_one 12)  ;;; 13
(abs_plus_one -12) ;;; -13
\end{lstlisting}

\section{ЗАДАНИЕ 3}

Написать функцию, которая принимает два числа и возвращает
список из этих чисел, расположенный по возрастанию.

\begin{lstlisting}
(defun sort_two_numbers (num1 num2)
    (if (< num1 num2)
        (list num1 num2)
        (list num2 num1)
    )
)

(sort_two_numbers 1 2) ;;; (1 2)
(sort_two_numbers 2 1) ;;; (1 2)
\end{lstlisting}

\section{ЗАДАНИЕ 4}

Написать функцию, которая принимает три числа и возвращает Т только
тогда, когда первое число расположенно между вторым и третьим.

\begin{lstlisting}
(defun between_two_numbers (num num1 num2)
    (cond
        ((and (> num num1) (< num num2)) T)
        ((and (< num num1) (> num num2)) T)
        (Nil)
    )
)

(between_two_numbers 1 2 3) ;;; Nil
(between_two_numbers 2 1 3) ;;; T
(between_two_numbers 2 3 1) ;;; T
\end{lstlisting}

\section{ЗАДАНИЕ 5}

Каков результат вычисления следующих выражений?

\begin{lstlisting}
(and 'fee 'fie 'foe)
;;; Результат: FOE

(or nil 'fie 'foe)
;;; Результат: FIE

(and (equal 'abc 'abc) 'yes)
;;; Результат: YES

(or 'fee 'fie 'foe)
;;; Результат: FEE

(and nil 'fie 'foe)
;;; Результат: NIL

(or (equal 'abc 'abc) 'yes)
;;; Результат: T
\end{lstlisting}

\section{ЗАДАНИЕ 6}

Написать предикат, который принимает два числа-аргумента и возвращает
Т, если первое число не меньше второго.

\begin{lstlisting}
(defun more_or_equal (num1 num2)
    (cond
        ((> num1 num2) T)
        ((equal num1 num2) T)
        (Nil)
    )
)

(more_or_equal 2 2) ;;; T
(more_or_equal 3 2) ;;; T
(more_or_equal 1 2) ;;; Nil
\end{lstlisting}

\section{ЗАДАНИЕ 7}

Какой из следующих двух вариантов предиката ошибочен и почему?

\begin{lstlisting}
(defun pred1 (x)
    (and (numberp x) (plusp x))
)

(defun pred2 (x)
    (and (plusp x) (numberp x))
)

(pred1 'a) ;;; Nil
(pred2 'a) ;;; Ошибка
\end{lstlisting}

Второй вариант ошибочен, так как сначала необходимо проверить,
является ли входной параметр числом, чтобы не возникло ошибки
при проверке аргумента на положительность.

\section{ЗАДАНИЕ 8}

Решить задачу 4, используя для ее решения конструкции
IF, COND, AND/OR.

\begin{lstlisting}
(defun between_two_numbers (num num1 num2)
    (if (or
            (and (< num num1) (> num num2))
            (and (> num num1) (< num num2))
        )
        T
        Nil
    )
)

(between_two_numbers 1 2 3) ;;; Nil
(between_two_numbers 2 1 3) ;;; T
(between_two_numbers 2 3 1) ;;; T
\end{lstlisting}
